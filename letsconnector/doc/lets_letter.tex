%% LyX 1.3 created this file.  For more info, see http://www.lyx.org/.
%% Do not edit unless you really know what you are doing.
\documentclass[10pt,a4paper,twocolumn,english,draft]{letter}
\usepackage{pslatex}
\usepackage[T1]{fontenc}
\usepackage[latin1]{inputenc}
\setlength\parskip{\medskipamount}
\setlength\parindent{0pt}
\IfFileExists{url.sty}{\usepackage{url}}
                      {\newcommand{\url}{\texttt}}

\makeatletter
%%%%%%%%%%%%%%%%%%%%%%%%%%%%%% User specified LaTeX commands.
\usepackage{a4wide}
\date{}

\usepackage{babel}
\makeatother
\begin{document}

\address{Edinburgh LETS\\
(A. Tullo \& K. Smallwood)\\
c/o Peace \& Justice Centre\\
St John's Church\\
Prince's Street\\
Edinburgh\\
EH2 4BJ\\
\\
Email: \texttt{\small sekenre@ukfsn.org}\\
Email: \texttt{\small alisdair@tullo.me.uk}~\\
{\small \url{http://www.edinburghlets.org.uk}}}


\letter{All LETS groups\\
The United Kingdom\\
North Atlantic\\
Earth}


\signature{Kio Smallwood\\
Alisdair Tullo}


\opening{Hello fellow LETS people!}

\begin{description}
\item [Is~your~committee]burning out due to all the paperwork of your
rapidly expanding LETSystem?
\item [Is~your~newsletter]constantly out of date, does it use a small
tree's worth of paper and a fortune in postage when you distribute
it?
\item [Do~your~members]use the Internet on a regular basis, and do they
use email to keep in touch with other members?
\end{description}
If the answer is yes to these questions then \textbf{LETS Connector}
could be for you. \textbf{LETS Connector} will be (when it's finished)
a complete system for running LETS over the Internet, it will sit
on a web-server%
\footnote{We can recommend a few good ones.%
} and members will be able to access it like a website. In operation
it will look and feel a bit like on-line banking, members will be
able to view account statements and make payments like LETS cheques.
It will also have a noticeboard for wants \& offers and a directory
where each member can keep their details up-to-date.

If you \emph{don't} think that this is for you, you are quite happy
to keep using paper records, and everyone in the committee is happy
with their workload, then \emph{please} forward a copy of this letter
to someone in your neighboring LETS groups. We can be fairly sure
that we don't have the address of every single LETS in the UK, and
this will ensure that we get maximum coverage. Thanks.

We are currently in the initial stages of development, we have written
a project plan and we are now ready to apply for development grants
from various community funding institutions. Before we really get
off the ground however, we would like to get a good idea of the level
of interest in this project within the UK so we can show potential
donors how many people would potentially benefit. We would also like
to hear from anyone in LETS' world-wide who would like to contribute
ideas, constructive criticism or would just like to be kept up-to-date
with current developments.

This project will be developed as Free Software%
\footnote{For a definition of Free Software see \url{http://www.gnu.org/philosophy/free-sw.html}%
} to ensure that the maximum number of LETS' will benefit. Free Software
in this case basically means that we will be sharing all of our development
work freely with everyone who is interested, and the whole project
can be downloaded and installed for free with few restrictions. The
restrictions are, that if you make changes to the project, you have
to share your changes under the same terms that we have, and you can't
pretend that you wrote it and then sell it on.

The following explains our project in a bit more detail ---

\begin{description}
\item [LETS~Connector]will be a LETS admin tool, accessible from the web
for the use of our members. It will be composed of an on-line directory,
accounting and payment system and an on-line newsletter. When this
project is complete members will be able to update their details,
post wants \& offers, view their account and carry out trades using
a secure website. 
\end{description}
We are doing this because currently, all of our admin is very centralised
with two of our committee members doing a disproportionate amount
of work, due to all of the paperwork that only they are equipped to
do.

A web based system has the following advantages ---

\begin{itemize}
\item It will allow LETS to expand faster as the workload for running the
system is decentralised.
\item A web based system allows individuals to access and change information
directly. In effect, members will `own' their information. Even those
not on the web will benefit because there will be more people who
are able to look up or change information for them.
\item A more robust system, as regular backups of the central database would
be made simpler and the data would not be held in someone's home.
(Vulnerable to theft, fire, etc\ldots{})
\item A more accessible system as access to it is not limited to the committee,
and the web connected members can provide info for non-connected members
as a service. (Also encourages more socialising between members.)
Free internet access in libraries \& community centres will provide
more access points.
\item The system will also provide accountability \& change control, so
that members can see who has changed or updated their details, and
administrators can keep track of large scale projects. If people can
see who changed what at the click of a button, then this will encourage
trustworthy behaviour.
\item Seeing all the current wants and offers together and knowing that
lots of trading is going on, can only encourage more trading. It might
also help new members to start trading sooner.
\item Most new members only begin trading after they have communicated with
other people in LETS. Usually this is at a trade fair. For those who
find trade fairs difficult---perhaps due to ill health, or low social
confidence---an internet-accessible LETS could be a good way to start.
\end{itemize}
Our next step is applying for funding so that we can spend a month
or two working on this full time, and have a working product relatively
quickly. We are applying to several different funding agencies including
UnLtd.%
\footnote{it used to be the Millennium Fund.%
} and the European Commission.

If your are a developer (or would like to \emph{learn} web-development)
who is interested in joining this project, and is familiar with (or
interested in) Python, MySQL and web-programming, you should have
a look at our development site at \url{http://developer.berlios.de/projects/letsconnector}
\emph{}and browse the CVS tree. You can then contact us, either through
the development website or our addresses above. 

We would like to hear from you for all enquiries and we will be compiling
a mailing list of everyone who would like regular updates on our progress.


\closing{Thanks,}


\cc{As many LETS' as possible \texttt{:-)}}
\end{document}
